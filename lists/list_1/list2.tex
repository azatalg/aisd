\documentclass[12pt]{article}

% ---------------------------------
%  PODSTAWOWE USTAWIENIA 
% ---------------------------------
\usepackage[a4paper,margin=2.5cm]{geometry}
\usepackage[T1]{fontenc}
\usepackage[utf8]{inputenc}

% ---------------------------------
%  FONTY: Fira Code (lub Fira Mono)
% ---------------------------------
% Jeśli masz zainstalowany pakiet FiraCode:
\usepackage{FiraCode}
% Jeżeli FiraCode nie działa, można użyć:
% \usepackage[scaled]{FiraMono}

% ---------------------------------
%  KOLORYSTYKA TOKYO NIGHT
% ---------------------------------
\usepackage{xcolor}
\definecolor{tokyoNightBG}{HTML}{1A1B26}    % Głęboki granatowo-fioletowy
\definecolor{tokyoNightFG}{HTML}{C0CAF5}    % Jasny, lekko fioletowy kolor tekstu
\definecolor{tokyoNightRed}{HTML}{F7768E}   % Różowy/czerwony odcień
\definecolor{tokyoNightGreen}{HTML}{9ECE6A} % Zieleń
\definecolor{tokyoNightBlue}{HTML}{7AA2F7}  % Niebieski
\definecolor{tokyoNightYellow}{HTML}{E0AF68}% Żółtawy/pomarańczowy

% Tło strony i kolor domyślny tekstu
\pagecolor{tokyoNightBG}
\color{tokyoNightFG}

% ---------------------------------
%  NAGŁÓWKI I STOPKI (fancyhdr)
% ---------------------------------
\usepackage{fancyhdr}
\pagestyle{fancy}
\fancyhf{}
\fancyhead[L]{\textcolor{tokyoNightGreen}{Algorithms & Wasted Souls}}
\fancyhead[R]{\textcolor{tokyoNightGreen}{\thepage}}
\renewcommand{\headrulewidth}{0pt}
\renewcommand{\footrulewidth}{0pt}

% ---------------------------------
%  STYL NAGŁÓWKÓW SEKCJI (titlesec)
% ---------------------------------
\usepackage{titlesec}
\titleformat{\section}
  {\normalfont\Large\bfseries\color{tokyoNightRed}}{\thesection.}{0.5em}{}
\titleformat{\subsection}
  {\normalfont\large\bfseries\color{tokyoNightBlue}}{\thesubsection.}{0.5em}{}
\titleformat{\subsubsection}
  {\normalfont\normalsize\bfseries\color{tokyoNightYellow}}{\thesubsubsection.}{0.5em}{}

% ---------------------------------
%  MINTED: KOLOROWANIE SKŁADNI
% ---------------------------------
\usepackage{minted}
% Wybieramy ciemny styl w miarę zbliżony do Tokyo Night:
\usemintedstyle{dracula}

% Definiujemy "środowisko" pseudokodu, używając wbudowanych reguł kolorowania C++
\newminted{cpp}{
  linenos,
  breaklines,
  frame=single,
  framesep=5pt,
  baselinestretch=1.0,
  fontsize=\small
}

% --- Tytuł i data ---
\title{\textbf{\huge Lista Zadań Nr 0 \\[0.3em]
{\Large Algorytmy i Struktury Danych}}}
\author{\textbf{\Large Wykładowca: Dr Obskurny}}
\date{\textbf{\Large \today}}  % automatyczna, bieżąca data kompilacji

\begin{document}

\maketitle

\vspace{2em}
\begin{center}
  \textit{\large „Neonowe zaułki nocnego Tokio mogą pomieścić i Big-O, i Big-$\Theta$, 
  ale czy pomieszczą naszą rozpacz?”}
\end{center}
\vspace{2em}

% ---------------------------------
%  TREŚĆ DOKUMENTU
% ---------------------------------

\section{Rozgrzewka (0 pkt)}
Przed Tobą kilka zadań. Odszukaj w sobie siłę, by je wypełnić, bo to tylko preludium 
do prawdziwego (algorytmicznego) mroku.

\section{Złożoności (1 pkt)}
Określ, z dokładnością do $\Theta$, złożoność następujących fragmentów (pseudokodu):

\begin{cppcode}
for (int i = 1; i <= n; i++) {
    int j = i;
    while (j < n) {
        sum = P(i, j);
        j = j + 1;
    }
}
\end{cppcode}

\begin{cppcode}
for (int i = 1; i <= n; i++) {
    int j = i;
    while (j < n) {
        sum = P(i, j);
        j = j + j;
    }
}
\end{cppcode}

Gdzie:
\begin{itemize}
  \item $P(i,j) = \Theta(1)$,
  \item $P(i,j) = \Theta(j)$.
\end{itemize}

\section{Szybkie potęgowanie (1 pkt)}
Pseudokod obliczający $x^n$ przez kolejne potęgi dwójkowe (dyskretnie ciesząc się 
logarytmiczną złożonością):

\begin{cppcode}
int fastPower(int x, int n) {
    int result = 1;
    int currentPower = x;
    int k = n;

    while (k > 0) {
        if (k % 2 == 1) {
            result = result * currentPower;
        }
        currentPower = currentPower * currentPower;
        k = k / 2;
    }
    return result;
}
\end{cppcode}

Rozważ złożoność w modelu jednorodnym i z uwzględnieniem kosztu mnożeń 
(czyli modelu logarytmicznego), by docenić, że nawet algorytmy mają 
drugie dno.

\section{Rekurencja w drzewach (1 pkt)}
Rekurencyjne liczenie liczby wierzchołków i średnicy w drzewie binarnym:

\begin{cppcode}
int countNodes(Node* T) {
    if (T == nullptr) 
        return 0;
    return 1 + countNodes(T->left) + countNodes(T->right);
}
\end{cppcode}

\begin{cppcode}
int treeDiameter(Node* T) {
    if (T == nullptr) 
        return 0;
    int leftHeight  = treeHeight(T->left);
    int rightHeight = treeHeight(T->right);

    // Średnica przechodząca przez bieżący węzeł
    int diameterThroughRoot = leftHeight + rightHeight;

    // Średnica w poddrzewach
    int leftDiameter  = treeDiameter(T->left);
    int rightDiameter = treeDiameter(T->right);

    return max(diameterThroughRoot, max(leftDiameter, rightDiameter));
}
\end{cppcode}

Zakładamy, że `treeHeight` zwraca wysokość drzewa.  
Tak, wciąż czekamy, aż odległość w drzewie zrówna się z odległością 
do zrozumienia sensu bytu.

\section{Operacje na drzewie BST (1 pkt)}
Wstawianie, usuwanie i znajdowanie następnika w drzewie BST (pseudokod inspirowany C++):

\begin{cppcode}
Node* insertBST(Node* T, int key) {
    if (T == nullptr) {
        T = new Node(key);
    } else if (key < T->value) {
        T->left = insertBST(T->left, key);
    } else {
        T->right = insertBST(T->right, key);
    }
    return T;
}
\end{cppcode}

\begin{cppcode}
Node* deleteBST(Node* T, int key) {
    if (T == nullptr) 
        return T;
    
    if (key < T->value) {
        T->left = deleteBST(T->left, key);
    } else if (key > T->value) {
        T->right = deleteBST(T->right, key);
    } else {
        // Znaleźliśmy węzeł do usunięcia
        if (T->left == nullptr) {
            Node* temp = T->right;
            delete T;
            return temp;
        } else if (T->right == nullptr) {
            Node* temp = T->left;
            delete T;
            return temp;
        } else {
            // Węzeł ma dwoje dzieci
            Node* successor = findMin(T->right);
            T->value = successor->value;
            T->right = deleteBST(T->right, successor->value);
        }
    }
    return T;
}
\end{cppcode}

\begin{cppcode}
Node* findSuccessor(Node* T, int key) {
    Node* successor = nullptr;
    Node* current = T;
    while (current != nullptr) {
        if (key < current->value) {
            successor = current;
            current = current->left;
        } else {
            current = current->right;
        }
    }
    return successor;
}
\end{cppcode}

(Czy to rozświetla mroki BST? Może tylko odrobinę – ale zawsze coś!)

\end{document}

